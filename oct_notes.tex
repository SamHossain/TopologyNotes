\documentclass[a4paper, 12pt]{article}
\newcommand{\RR}{\mathbb{R}}
\newcommand{\XX}{\mathbf{X}}
\newcommand{\YY}{\mathbf{Y}}
\newcommand{\QQ}{\mathbb{Q}}
\newcommand{\KK}{\mathbf{K}}
\newcommand{\CC}{\mathbf{C}}

\usepackage{amsmath,inputenc, amsthm, amssymb, enumitem}

\newtheorem{theorm}{Theorem}

\begin{document}

\title{October Notes}
\author{Sam Hossain}

\maketitle
\section{Hausdorff Spaces}
\begin{theorm}
Let $\XX$ be a hausdorff space. Every finite subset is closed.
\end{theorm}
\begin{proof}
Recall that finite unions of closed subsets are closed. So it suffices to show that everyt singleton is closed. Lets fix $x \in \XX$, now for every $y \in \XX \setminus {x} \exists$ nneighborhood $U_y$ of x and $V_y$ of y such that $U_y \cap V_y = \emptyset$. We claim that $\bigcup\limits_{y \in \XX \setminus {x}} V_y = \XX \setminus {x}$. But we know $V_y$ is open so their union is open which means $\XX \setminus {x}$ is open $\Rightarrow {x}$ closed.
\end{proof}

\begin{theorm}
Let $\XX$ be a hausdorff space. $A \subset \XX$ with a limit point, x. $\Leftrightarrow$ ever neighborhood of $\XX$ contains infinitely many points of A.
\end{theorm}
\begin{proof}~
        \begin{enumerate}
                \item[$\Rightarrow$] Suppose that a neighborhood of x, $U_x$ untersects A at finitely many points. Let $(U_x - {x}) \cap A = {x_1, \dots, x_n}$. This is closed so $U_x - {x_1, \dots, x_n}$ is open, which is also a neighborhood that intersects A at {x} at most.
                \item[$\Leftarrow$] This follows from the definition (i.e. if there are infinitely many points, one isn't x)
        \end{enumerate}
\end{proof}


\section{Product Topology}   
Before we introduce the product top. on $\RR^{\omega}$, lets motivate it with the Box Topology. 
\newline
\textbf{Definition:} Let $\prod\limits_{\alpha \in J} X_{\alpha}$ be a topological space with $U_{\alpha}$ be an open set in $X_{\alpha}$. Then let $\prod\limits_{\alpha \in J} U_{\alpha}$ be a basis for a topology of that space. The topology generated by this basis is the Box Topology.
\newline
Note that this space does not behave well. Consider: $\RR^{\infty}$ and the set of points, $(1, 1 \dots) = 1^{\infty}, (\frac{1}{2})^{\infty},(\frac{1}{3})^{\infty}, \dots, (\frac{1}{n})^{\infty}, \dots$. We would expect this to converge to $(0)^{\infty}$ but it doesn't in the Box top. Consider $(-1, 1) \times (-\frac{1}{2}, \frac{1}{2}) \times \dots = S$. Clearly, $0 \in S$ but our set of points as a sequence will not be contained in this open set for any finite n. Thus, does not converge. \newline
\textbf{Definition:} Let $U_i$ be an open subset of the $i^{th}$ component of an infinite product, $\prod\limits_{\alpha \in J} X_{\alpha}$. If we take finitely many of these $U_i$ along with the rest of the compononts as basis elements. That is, $\mathbb{B} = \{\prod\limits_{\alpha \in J} X_{\alpha} | U_{\alpha} \subseteq X_{\alpha} \text{ open, and } U_{\alpha} = X_{\alpha} \text{ for all but finitely many } \alpha \}$ generate a basis for a topology on $\prod\limits_{\alpha \in J} X_{\alpha}$. This is the product topology.
\begin{theorm}
        Let $A_{\alpha}$ be a subspace of$ X_{\alpha}$. Then $\prod\limits_{\alpha \in J} A_{\alpha}$ is equal to the subspace topology of $\prod\limits_{\alpha \in J} X_{\alpha}$ if both products are given by the box topology, on both products given by product topology.
\end{theorm}

\section{Quotient Topology}
\textbf{Definition:} Let $\XX$ and $\YY$ be topological spaces with a continous map $p: \XX \rightarrow \YY$ is called a quotient map if p is surjective and $U \subset \YY$ open $\Leftrightarrow$ $p^{-1}(U) \subset \XX$ is open.\newline
\textbf{Definition:} If $\XX$ is a topological space, A is a set, and $p: \XX \rightarrow A$ is a surjective map. Then there exista a unique topology on A such that p is a quotient map. This topology on A is called the quotient topology. \textbf{Remark:} This is the finest topology on A.\newline
\textbf{Definition:} Let $\sim$ be an equivalence relation on X. Then $X /\sim$ is the set on equivalence classes on X and $X \rightarrow X /\sim$ is a surjective map.\newline



\end{document}
